\documentclass{article}
\setlength{\parskip}{5pt} % esp. entre parrafos
\setlength{\parindent}{0pt} % esp. al inicio de un parrafo
\usepackage{amsmath} % mates
\usepackage[sort&compress,numbers]{natbib} % referencias
\usepackage{url} % que las URLs se vean lindos
\usepackage[top=25mm,left=20mm,right=20mm,bottom=25mm]{geometry} % margenes
\usepackage{hyperref} % ligas de URLs
\usepackage{graphicx} % poner figuras
\usepackage[spanish]{babel} % otros idiomas
\usepackage[utf8]{inputenc}
\author{Alfredo Cárdenas Mena \\
Daniel Garcia Rodarte\\
Angel Eduardo Gonzalez Melendres \\
Angel Mario Alcala Rodriguez \\
Aarón Lozano Aguilar} % author
\title{Biomecanica de la Mano} % titulo
\date{\today}
\usepackage{float} 
\begin{document} % inicia contenido

\maketitle % cabecera

\Title{Ensayo}\label{Title} % seccion y etiqueta
Al momento de abrir los dedos de la mano lo más que se pueda, los ejes las líneas de los ejes pasan por un mismo punto haciendo que estas se unan tanto que se puede apreciar la separación de los dedos, las líneas imaginarias de los ejes de los dedos, etc.\\
Al realizar el cuarto movimiento de la mano con los dedos, cerrar el puno, los ejes convergen un punto situado en la base del talón de la mano, la flexión activa que se tiene en esta articulación es de casi 90 grados en el índice aumentando de manera progresiva hasta el dedo menique cuando todos los dedos se flexionan.
El dedo donde es el que tiene mayor amplitud en los movimientos de abducción y aducción, estos movimientos pueden llegar hasta los 30 grados, sin depender de los demás dedos.
Los músculos de los tendones flexores de los dedos inician en el epitroclear humeral y van hacia la cara palmar.
El flexor común profundo de los dedos está en la base de la tercera falange luego de perforar al flexor común superficial que se divide en dos lengüetas.
Comúnmente se piensa que al terminar la segunda falange sería más sencillo, sin embargo, según el punto de vista biomecánico menciona que el ´ángulo de tracción o ´ángulo de acercamiento es mayor en esta disposición anatómica que si cada uno estuviera en contacto directo con el esqueleto.\\
El flexor común superficial de los dedos es flexor de la segunda falange porque esta insertado en las caras laterales de ella, como consecuencia no tiene acción sobre la tercera falange y sobre la primera falange actúa solo cuando la segunda está completamente flexionada.
Además, obtiene potencia cuando la primera falange está en extensión por la acción del extensor común de los dedos.\\
La potencia máxima se adquiere cuando la primera falange está en extensión por contracción del extensor común de los dedos.
Los músculos de los tendones extensores de los dedos se originan en el epicóndilo humeral y terminan en la cara dorsal y se localizan al nivel de la muñeca y por debajo del ligamento anular posterior del carpo.
El extensor común de los dedos solo se involucra en la primera falange sobre el metacarpiano, se realiza cuando él esta se expande para insertarse en la base de la primera falange.
Son una pieza clave para hacer movimientos de lateralidad y de flexo extensión de los dedos.\\
Estos movimientos dependen de la dirección del cuerpo muscular entonces cuando se dirige al eje de la mano los intereses dorsales son los responsables de la separación de los dedos y cuando se aleja del eje de la mano los ´óseos palmares determinaran la aproximación de los dedos.
Cuando la articulación metacarpofalángica está en extensión la cubierta dorsal de los Inter óseos se sitúa en el dorso del cuello del primer metacarpiano, así los músculos interóseos pueden tensar las expansiones laterales y con esto extender la segunda y la tercera falange.\\
Los músculos lumbricales son músculos intrínsecos de la mano que son fundamentales en los movimientos de flexo extensión de los dedos, esto gracias a que está ubicado en un plano palmar que le deja contar con un ´ángulo de incidencia de 35° con respecto a la primera falange dejando les flexionarse, aunque este en hiperextensión, de igual manera tiene extensión de la segunda y tercera falanges sin importar cual sea el grado de flexión de la articulación metacarpofalángica.\\
El ligamento retinacular es el que se encuentra a cada lado de la articulación interfalángica proximal y se inserta en la cara palmar de la primera falange.
Este ligamento se dirige hacia las cintillas laterales del extensor común en el dorso de la segunda falange.
El flexor corto del menique flexiona la primera falange sobre el primer metacarpiano.\\
El aductor del menique tiene la misma acción que el flexor corto, son por tanto abductores del dedo menique con respecto al eje de la mano.
Son ademases flexores de la primera falange y extensores de la segunda y tercera en una sección semejante a la de los interóseos dorsales.
El primer metacarpiano es el más corto de la mano y presenta un cuerpo más aplanado en sentido dorso palmar que los restantes, representa la primera falange de los otros dedos.\\
Esta articulación de anclaje reciproco permite al pulgar orientarse en relaciona con el resto de la mano en todos los planos del espacio.
\\\\
Realiza dos movimientos: Movimiento de antepulsión y retropulsión & Movimientos de aducción y abducción
\\\\
Es una articulación de tipo con, permite dos tipos de movimientos, lo que le confiere una gran importancia ya que son movimientos no muy habituales en las articulaciones con estas características.
Los movimientos de lateralidad no existen en esta articulación, pero su ausencia está ampliamente compensada por la gran movilidad de la articulación trapecio metacarpiana que permite todos los movimientos en el espacio, siendo el primer metacarpiano el gran beneficiado al estar el trapecio inmovilizado en la segunda hilera del carpo.\\
Esta articulación es de tipo troclear como cualquiera de las articulaciones interfalángicas, solo permite movimientos de flexo extensión.
El extensor corto del pulgar hace la extensión de la primera Falange, pero este lleva a los dos directamente hacia afuera por lo que este se convierte en el principal abductor del pulgar, el flexor largo propio del pulgar es en realidad el flexor de la tercera Falange sobre la primera, desempeña un papel muy importante en este movimiento.\\
El abductor corto del pulgar tiene varias funciones los cuales son aducción, y antepulsión del primer metacarpiano sobre el Carpio, flexor de la primera Falange sobre la Falange con inclinación externa y rotación axial.
El principal trabajo de la mano es la prensa, la mano se auxilia del dedo pulgar para realizar la acción de pinza potente ya que este dedo se opone al resto de dedos, esto aunado con los movimientos coordinados facilita bastante esta acción.\\
Las funcionen se dividen en manera en que utilizamos nuestros dedos, la fuerza aplicada y los músculos que interactúan de distinta manera para llegar a poder adaptarse a lo que sea que se esté haciendo.




\end{document}